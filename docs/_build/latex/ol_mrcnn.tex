%% Generated by Sphinx.
\def\sphinxdocclass{report}
\documentclass[letterpaper,10pt,english]{sphinxmanual}
\ifdefined\pdfpxdimen
   \let\sphinxpxdimen\pdfpxdimen\else\newdimen\sphinxpxdimen
\fi \sphinxpxdimen=.75bp\relax
\ifdefined\pdfimageresolution
    \pdfimageresolution= \numexpr \dimexpr1in\relax/\sphinxpxdimen\relax
\fi
%% let collapsible pdf bookmarks panel have high depth per default
\PassOptionsToPackage{bookmarksdepth=5}{hyperref}

\PassOptionsToPackage{booktabs}{sphinx}
\PassOptionsToPackage{colorrows}{sphinx}

\PassOptionsToPackage{warn}{textcomp}
\usepackage[utf8]{inputenc}
\ifdefined\DeclareUnicodeCharacter
% support both utf8 and utf8x syntaxes
  \ifdefined\DeclareUnicodeCharacterAsOptional
    \def\sphinxDUC#1{\DeclareUnicodeCharacter{"#1}}
  \else
    \let\sphinxDUC\DeclareUnicodeCharacter
  \fi
  \sphinxDUC{00A0}{\nobreakspace}
  \sphinxDUC{2500}{\sphinxunichar{2500}}
  \sphinxDUC{2502}{\sphinxunichar{2502}}
  \sphinxDUC{2514}{\sphinxunichar{2514}}
  \sphinxDUC{251C}{\sphinxunichar{251C}}
  \sphinxDUC{2572}{\textbackslash}
\fi
\usepackage{cmap}
\usepackage[T1]{fontenc}
\usepackage{amsmath,amssymb,amstext}
\usepackage{babel}



\usepackage{tgtermes}
\usepackage{tgheros}
\renewcommand{\ttdefault}{txtt}



\usepackage[Bjarne]{fncychap}
\usepackage{sphinx}

\fvset{fontsize=auto}
\usepackage{geometry}


% Include hyperref last.
\usepackage{hyperref}
% Fix anchor placement for figures with captions.
\usepackage{hypcap}% it must be loaded after hyperref.
% Set up styles of URL: it should be placed after hyperref.
\urlstyle{same}


\usepackage{sphinxmessages}




\title{OL\_mrcnn}
\date{Jun 03, 2024}
\release{1.0.0}
\author{Tom Soulaire}
\newcommand{\sphinxlogo}{\vbox{}}
\renewcommand{\releasename}{Release}
\makeindex
\begin{document}

\ifdefined\shorthandoff
  \ifnum\catcode`\=\string=\active\shorthandoff{=}\fi
  \ifnum\catcode`\"=\active\shorthandoff{"}\fi
\fi

\pagestyle{empty}
\sphinxmaketitle
\pagestyle{plain}
\sphinxtableofcontents
\pagestyle{normal}
\phantomsection\label{\detokenize{index::doc}}


\sphinxAtStartPar
This tool is a model based on a MRCNN architecture that enables to
\begin{itemize}
\item {} 
\sphinxAtStartPar
detect and crop cells in an image (grayscale, RGB, or more channels)

\item {} 
\sphinxAtStartPar
classify cells in an image

\item {} 
\sphinxAtStartPar
segment cells in an image

\end{itemize}


\chapter{Installation}
\label{\detokenize{index:installation}}
\sphinxAtStartPar
To install the model on a local machine (requiring Python 3.9):
\begin{enumerate}
\sphinxsetlistlabels{\arabic}{enumi}{enumii}{}{.}%
\item {} 
\sphinxAtStartPar
create conda environment:

\end{enumerate}

\begin{sphinxVerbatim}[commandchars=\\\{\}]
\PYG{n}{conda} \PYG{n}{env} \PYG{n}{create} \PYG{o}{\PYGZhy{}}\PYG{n}{f} \PYG{n}{environment}\PYG{o}{.}\PYG{n}{yml}
\end{sphinxVerbatim}
\begin{enumerate}
\sphinxsetlistlabels{\arabic}{enumi}{enumii}{}{.}%
\setcounter{enumi}{1}
\item {} 
\sphinxAtStartPar
activate conda environment:

\end{enumerate}

\begin{sphinxVerbatim}[commandchars=\\\{\}]
\PYG{n}{conda} \PYG{n}{activate} \PYG{n}{OL\PYGZus{}mrcnn}
\end{sphinxVerbatim}
\begin{enumerate}
\sphinxsetlistlabels{\arabic}{enumi}{enumii}{}{.}%
\setcounter{enumi}{2}
\item {} 
\sphinxAtStartPar
download the model weights \sphinxhref{https://drive.google.com/drive/folders/1PIstT451WQIOS59vtHqkq8PTD-xO0Gj\_?usp=sharing}{here} and place them in the folder \sphinxcode{\sphinxupquote{/logs}}. These contain the original weights (trained on COCO dataset) and the weights trained on a custom dataset.

\end{enumerate}


\chapter{User Guide}
\label{\detokenize{index:user-guide}}

\section{Image cropping}
\label{\detokenize{index:image-cropping}}\begin{itemize}
\item {} 
\sphinxAtStartPar
Add your dataset in the folder \sphinxcode{\sphinxupquote{/data}}

\item {} 
\sphinxAtStartPar
OPTIONAL: preprocess your data with the \sphinxcode{\sphinxupquote{preprocessing.ipynb}} notebook

\item {} 
\sphinxAtStartPar
Configure the \sphinxcode{\sphinxupquote{image\_cropper.ipynb}} notebook:
\begin{itemize}
\item {} 
\sphinxAtStartPar
\sphinxcode{\sphinxupquote{DEVICE}}: device to use for inference. Default value is ‘cpu:0’.

\item {} 
\sphinxAtStartPar
\sphinxcode{\sphinxupquote{detection\_min\_confidence}}: minimum confidence level for the detections. Default value is 0.7.

\item {} 
\sphinxAtStartPar
\sphinxcode{\sphinxupquote{detection\_nms\_threshold}}: non\sphinxhyphen{}maximum suppression threshold. Eliminates the least confident detection when the IoU of 2 masks is above this value. Default value is 0.3.

\item {} 
\sphinxAtStartPar
\sphinxcode{\sphinxupquote{weights\_subpath}}: subpath in the \sphinxtitleref{/logs} folder to the weights file.

\item {} 
\sphinxAtStartPar
\sphinxcode{\sphinxupquote{results\_name}}: name of the folder where the results will be saved.

\item {} 
\sphinxAtStartPar
\sphinxcode{\sphinxupquote{test\_dir}}: name of the folder where the images are stored.

\item {} 
\sphinxAtStartPar
\sphinxcode{\sphinxupquote{num\_gpu}}: number of GPUs to use for inference. Default value is 1.

\item {} 
\sphinxAtStartPar
\sphinxcode{\sphinxupquote{num\_img\_per\_gpu}}: number of images to process in parallel on each GPU. Default value is 1.

\item {} 
\sphinxAtStartPar
\sphinxcode{\sphinxupquote{VISUALIZE}}: if True, displays the images with the detections. Default value is False.

\end{itemize}

\item {} 
\sphinxAtStartPar
Run the notebook. The results will be saved in the folder \sphinxcode{\sphinxupquote{/results/results\_name}}.

\end{itemize}


\section{Full pipeline}
\label{\detokenize{index:full-pipeline}}\begin{itemize}
\item {} 
\sphinxAtStartPar
In this setup, we run a first model to crop and classify objects in the images. Then we run a second model on the cropped images to get a refined mask.

\item {} 
\sphinxAtStartPar
In the \sphinxcode{\sphinxupquote{model\_pipeline.ipynb}} notebook, configure the following parameters:
\begin{itemize}
\item {} 
\sphinxAtStartPar
\sphinxcode{\sphinxupquote{DEVICE}}: device to use for inference. Default value is ‘cpu:0’.

\item {} 
\sphinxAtStartPar
\sphinxcode{\sphinxupquote{gpu\_count\_macro}}: number of GPUs to use for the first model. Default value is 1.

\item {} 
\sphinxAtStartPar
\sphinxcode{\sphinxupquote{num\_img\_per\_gpu\_macro}}: number of images to process in parallel on each GPU for the first model. Default value is 1.

\item {} 
\sphinxAtStartPar
\sphinxcode{\sphinxupquote{min\_confidence\_macro}}: minimum confidence level for the detections in the first model. Default value is 0.7.

\item {} 
\sphinxAtStartPar
\sphinxcode{\sphinxupquote{nms\_threshold\_macro}}: non\sphinxhyphen{}maximum suppression threshold for the first model. Default value is 0.3.

\item {} 
\sphinxAtStartPar
\sphinxcode{\sphinxupquote{nms\_multiclass\_macro}}: non\sphinxhyphen{}maximum suppression threshold between classes for the first model. Default value is 0.3.

\item {} 
\sphinxAtStartPar
\sphinxcode{\sphinxupquote{gpu\_count\_micro}}: number of GPUs to use for the second model. Default value is 1.

\item {} 
\sphinxAtStartPar
\sphinxcode{\sphinxupquote{num\_img\_per\_gpu\_micro}}: number of images to process in parallel on each GPU for the second model. Default value is 1.

\item {} 
\sphinxAtStartPar
\sphinxcode{\sphinxupquote{min\_confidence\_micro}}: minimum confidence level for the detections in the second model. Default value is 0.7.

\item {} 
\sphinxAtStartPar
\sphinxcode{\sphinxupquote{nms\_threshold\_micro}}: non\sphinxhyphen{}maximum suppression threshold for the second model. Default value is 0.3.

\item {} 
\sphinxAtStartPar
\sphinxcode{\sphinxupquote{MACRO\_MODEL\_SUBPATH}}: subpath in the \sphinxtitleref{/logs} folder to the weights file of the first model.

\item {} 
\sphinxAtStartPar
\sphinxcode{\sphinxupquote{MICRO\_MODEL\_SUBPATH}}: subpath in the \sphinxtitleref{/logs} folder to the weights file of the second model.

\item {} 
\sphinxAtStartPar
\sphinxcode{\sphinxupquote{RESULTS\_NAME}}: name of the folder where the results will be saved.

\item {} 
\sphinxAtStartPar
\sphinxcode{\sphinxupquote{TEST\_DIR}}: name of the folder where the images are stored.

\item {} 
\sphinxAtStartPar
\sphinxcode{\sphinxupquote{VISUALIZE}}: if True, displays the images with the detections. Default value is False.

\end{itemize}

\item {} 
\sphinxAtStartPar
Run the notebook. The results will be saved in the folder \sphinxcode{\sphinxupquote{/results/RESULTS\_NAME}}.

\end{itemize}


\section{Retraining your own model}
\label{\detokenize{index:retraining-your-own-model}}

\subsection{Data structure}
\label{\detokenize{index:data-structure}}\begin{itemize}
\item {} 
\sphinxAtStartPar
Create a \sphinxcode{\sphinxupquote{/data}} folder in the root directory.

\item {} 
\sphinxAtStartPar
Inside the \sphinxcode{\sphinxupquote{/data}} directory, put your images in a folder named \sphinxcode{\sphinxupquote{/imgs}} and your binary masks in a folder named \sphinxcode{\sphinxupquote{/masks}}. The name, size and format of the masks must match the images.

\item {} 
\sphinxAtStartPar
In the {\color{red}\bfseries{}\textasciigrave{}\textasciigrave{}}roi\_labels\_to\_json.py\textasciigrave{}\textasciigrave{}script, configure the {\color{red}\bfseries{}\textasciigrave{}\textasciigrave{}}dir\_path\textasciigrave{}\textasciigrave{}in the \sphinxtitleref{main()} function. Run in a terminal:

\end{itemize}

\begin{sphinxVerbatim}[commandchars=\\\{\}]
\PYG{n}{python} \PYG{n}{roi\PYGZus{}labels\PYGZus{}to\PYGZus{}json}\PYG{o}{.}\PYG{n}{py}
\end{sphinxVerbatim}
\begin{itemize}
\item {} 
\sphinxAtStartPar
Move the label files to a \sphinxcode{\sphinxupquote{jsons}} folder in the {\color{red}\bfseries{}\textasciigrave{}\textasciigrave{}}/data\textasciigrave{}\textasciigrave{}directory.

\item {} 
\sphinxAtStartPar
In the \sphinxcode{\sphinxupquote{format\_data.py}} script, configure the \sphinxcode{\sphinxupquote{dir\_path}} in the \sphinxtitleref{main()} function. Configure the size the of the training / validation / test datasets (usually 0.6, 0.2, 0.2) Run in a terminal:

\end{itemize}

\begin{sphinxVerbatim}[commandchars=\\\{\}]
\PYG{n}{python} \PYG{n}{format\PYGZus{}data}\PYG{o}{.}\PYG{n}{py}
\end{sphinxVerbatim}


\subsection{Retraining a single class model}
\label{\detokenize{index:retraining-a-single-class-model}}\begin{itemize}
\item {} \begin{description}
\sphinxlineitem{In the \sphinxcode{\sphinxupquote{custom.py}} script, configure the following:}\begin{itemize}
\item {} 
\sphinxAtStartPar
\sphinxcode{\sphinxupquote{GRAYSCALE}}: if True, the model will be trained on grayscale images. Default value is False.

\item {} 
\sphinxAtStartPar
\sphinxcode{\sphinxupquote{DATA\_PATH}}: path to the dataset. Default value is ‘/data’.

\item {} 
\sphinxAtStartPar
\sphinxcode{\sphinxupquote{NAME}}: name of the model.

\item {} 
\sphinxAtStartPar
\sphinxcode{\sphinxupquote{GPU\_COUNT}}: number of GPUs to use. Default value is 1.

\item {} 
\sphinxAtStartPar
\sphinxcode{\sphinxupquote{IMAGES\_PER\_GPU}}: number of images to process in parallel on each GPU. Default value is 1.

\item {} 
\sphinxAtStartPar
\sphinxcode{\sphinxupquote{NUM\_CLASSES}}: number of classes. Default value is 2.

\item {} 
\sphinxAtStartPar
\sphinxcode{\sphinxupquote{EPOCHS}}: number of epochs. Default value is 50.

\item {} 
\sphinxAtStartPar
\sphinxcode{\sphinxupquote{STEPS PER EPOCH}}: number of steps per epoch. Default value is 50.

\item {} 
\sphinxAtStartPar
\sphinxcode{\sphinxupquote{LEARNING\_RATE}}: learning rate. Default value is 0.001.

\item {} 
\sphinxAtStartPar
\sphinxcode{\sphinxupquote{LAYERS}}: layers to train. Default value is ‘heads’.

\item {} 
\sphinxAtStartPar
\sphinxcode{\sphinxupquote{DETECTION\_MIN\_CONFIDENCE}}: minimum confidence level for the detections. Default value is 0.7.

\item {} 
\sphinxAtStartPar
\sphinxcode{\sphinxupquote{DEVICE}}: device to use for training. Default value is ‘cpu:0’.

\item {} 
\sphinxAtStartPar
\sphinxcode{\sphinxupquote{MAX\_GT\_INSTANCES}}: maximum number of instances in the ground truth. Default value is 100.

\item {} 
\sphinxAtStartPar
\sphinxcode{\sphinxupquote{DETECTION\_MAX\_INSTANCES}}: maximum number of instances in the detections. Default value is 35.

\item {} 
\sphinxAtStartPar
in the {\color{red}\bfseries{}\textasciigrave{}\textasciigrave{}}CustomDataset\textasciigrave{}\textasciigrave{}class, modify or add lines :

\end{itemize}

\end{description}

\item {} 
\sphinxAtStartPar
Run the script in a terminal:

\end{itemize}

\begin{sphinxVerbatim}[commandchars=\\\{\}]
\PYG{n}{python} \PYG{n}{custom}\PYG{o}{.}\PYG{n}{py}
\end{sphinxVerbatim}


\subsection{Retraining a multi\sphinxhyphen{}class model}
\label{\detokenize{index:retraining-a-multi-class-model}}\begin{itemize}
\item {} 
\sphinxAtStartPar
Same instructions as before but on the \sphinxcode{\sphinxupquote{custom\_multi.py}} script.

\item {} 
\sphinxAtStartPar
Run the script in a terminal:

\end{itemize}

\begin{sphinxVerbatim}[commandchars=\\\{\}]
\PYG{n}{python} \PYG{n}{custom\PYGZus{}multi}\PYG{o}{.}\PYG{n}{py}
\end{sphinxVerbatim}


\chapter{Indices and tables}
\label{\detokenize{index:indices-and-tables}}\begin{itemize}
\item {} 
\sphinxAtStartPar
\DUrole{xref,std,std-ref}{genindex}

\item {} 
\sphinxAtStartPar
\DUrole{xref,std,std-ref}{modindex}

\item {} 
\sphinxAtStartPar
\DUrole{xref,std,std-ref}{search}

\end{itemize}



\renewcommand{\indexname}{Index}
\printindex
\end{document}